\documentclass{article}
\usepackage[spanish]{babel}
\usepackage{graphicx}
\usepackage{xcolor}
\usepackage[utf8]{inputenc}
\usepackage{fancyhdr}
\usepackage{lastpage}
\usepackage{enumitem}
\usepackage{listings}
\usepackage{float}

\pagestyle{fancy}
\fancyhf{}
\rfoot{Page \thepage\hspace{1pt} de~\pageref{LastPage}}

\title{Tarea 3}
\author{Guillermo López García}
\begin{document}

\textbf{1º Ejercicio.}
\begin{enumerate}
    \item \underline{Comunicación Síncrona}:
        \begin{enumerate}
            \item 1º Escenario: Modificación de usuarios de una base de datos mediante un programa cliente y un programa servidor.
            \item 2º Escenario: Autentificación de login en un sistema software.
        \end{enumerate}
    \item \underline{Comunicación Asíncrona}:
        \begin{enumerate}
            \item 1º Escenario: Consulta de usuarios de una base de datos para mostrar en una tabla.
            \item 2º Escenario: Cualquier sistema que se comunique mediante el protocolo UDP, como por ejemplo, el Fortnite.
        \end{enumerate}
\end{enumerate}

\textbf{2º Ejercicio.}
\\
No, ya que, si varios receptores de distinta indole se encuentra escuchando en un mismo
puerto, puede provocar colision y perdida de información.\\
Si por el contrario, son varios receptores del mismo servicio, entonces si, ya que aumenta
la posibilidad de atender el servicio y la no saturación de un receptor y que en consecuencia,
se cree un Denice of Service (D-DOS).\\

\textbf{3º Ejercicio.}
\begin{enumerate}
    \item El cliente sabe como localizar el servicio mediante un IdInvocacion, en el cual se
          setea automáticamente dirección IP origen y destino, y puerto donde se localiza el
          servicio.
    \item En principio, cada puerto tiene asociado un identificador numerico único, asi pues,
          la eficiencia de acceso es de O (1). Por otra parte, el identificador local es el
          identificador numerico único descrito anteriormente y se utiliza para localizar el
          puerto y además, poder crear sockets sobre él para poder escuchar las peticiones
          recibidas por él.
\end{enumerate}

\end{document}
