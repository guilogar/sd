\documentclass{article}
\usepackage[spanish]{babel}
\usepackage{graphicx}
\usepackage{xcolor}
\usepackage[utf8]{inputenc}
\usepackage{fancyhdr}
\usepackage{lastpage}
\usepackage{enumitem}
\usepackage{listings}
\usepackage{verbatim}
\usepackage{float}

\pagestyle{fancy}
\fancyhf{}
\rfoot{Page \thepage\hspace{1pt} de~\pageref{LastPage}}

\title{Práctica 3}
\author{
Guillermo López García
\and
Guillermo Girón García
\and
Teodoro Martínez \ldots
}

\begin{document}
\maketitle
\textbf{1º Parte. Descripción de la actividad a realizar.}\\
Nuestro sistema se basa en informar de toda actividad de un usuario
en concreto de github a la plataforma de red social Twitter, y si es
posible, hacer una copia de seguridad de los ficheros afectados
del repositorio donde el usuario ha hecho los cambios que ha considerado
oportunos, en google drive, en dropbox o en cualquier otra plataforma
de almacenamiento oportuno.

\textbf{2º Parte. Descripción de las partes a realizar.}\\
La actividad se ha descompuesto en 3 partes bien diferenciadas.
A continuación, las enumeramos.
\begin{enumerate}
    \item \underline{1º Parte (Guillermo López García)}: Esta parte se
        basa en la obtención de la información del usuario de GitHub mediante
        las claves de aplicación OAuth creada previamente en GitHub.
        Claro esta, este usuario debe proporcionar estas claves.
        A su vez, este programa se ejecuta cada cierto tiempo, haciendo spooling,
        y enviando a una cola del sistema rabbbitmq la información que haya cambiado.
        También, es necesario aclarar que el programa compara siempre la información,
        con otra almacenada previamente en una base de datos PostgreSQL\. Si la
        información cambia, el programa la refresca en la base de datos y, como se
        ha dicho antes, la introduce en forma de JSON en una cola del sistema
        rabbitmq.
    \item \underline{2º Parte (Guillermo Girón García)}: Consiste en obtener la información
        contenida en una cola de rabbitmq mediante spooling, separarla en los campos
        necesarios, después de un "loggin" en Twitter, publicar un pequeño mensaje en 
        la red social que indique que se ha realizado un commit en la cuenta de GitHub del
        usuario, y permita enlazar al mismo, para observar todos los cambios y el mensaje del
        commit. 
    \item \underline{3º Parte (Teodoro Martínez \ldots)}:\ldots\ldots\ldots\ldots\ldots
\end{enumerate}

\textbf{3º Parte. Descripción de las tecnologías usadas.}\\
Las tecnologías usadas son las siguientes:
\begin{enumerate}
    \item \underline{1º Parte (Guillermo López García)}: En esta parte, se ha usado
        la base de datos relacional PostgreSQL, el lenguaje interpretado de servidor
        PHP, la librería Milo/GitHub para obtener los datos del usuario de GitHub
        (es posible su instalación mediante composer) y la librería PhpAmqpLib para
        poder realizar acciones sobre las colas del sistema rabbitmq (es posible su
        instalación mediante composer). Además, la versión del servidor apache es la
        2.2, la versión de PostgreSQL es la 9.5 y del interprete PHP es la 7.2
    \item \underline{2º Parte (Guillermo Girón García)}: Se ha utilizado Python 3 como
    lenguaje de programación para la implementación, con las siguientes bibliotecas:
    \begin{enumerate}
        \item Pika, para trabajar de forma sencilla con las colas de rabbitmq. 
        \item Python-Twitter, para acceder de manera sencilla a la api de Twitter. 
        \item La carga de los credenciales de las cuentas de twitter y dropbox se realizan desde un fichero Json.
    \end{enumerate} 
    \item \underline{3º Parte (Teodoro Martínez \ldots)}:\ldots\ldots\ldots\ldots\ldots
\end{enumerate}

\textbf{4º Parte. Código usado para realizar el programa.}
\begin{enumerate}
    \item \underline{1º Parte (Guillermo López García)}:
        \begin{enumerate}
            \item \underline{Base de Datos PostgreSQL}:
                \lstset{
                  language=SQL,
                  texcl=true,
                  basicstyle=\ttfamily,
                  columns=fullflexible,
                  frame=single,
                  breaklines=true,
                  postbreak=\mbox{\textcolor{red}{$\hookrightarrow$}\space},
                }
                \lstinputlisting[]{sql.sql}
            \item \underline{Código de Servidor PHP}:
                \lstset{
                  texcl=true,
                  basicstyle=\ttfamily,
                  columns=fullflexible,
                  frame=single,
                  breaklines=true,
                  postbreak=\mbox{\textcolor{red}{$\hookrightarrow$}\space},
                }
                \lstinputlisting[]{spooling_github.php}
            \item \underline{composer.json}:
                \lstinputlisting[]{composer.json}
        \end{enumerate}
    \item \underline{2º Parte (Guillermo Girón García)}:\ldots\ldots\ldots\ldots\ldots
    \item \underline{3º Parte (Teodoro Martínez \ldots)}:\ldots\ldots\ldots\ldots\ldots
\end{enumerate}

\textbf{5º Parte. Configuración del entorno.}
Como se habrá podido comprobar, en el codigo PHP anteriormente expuesto, se
han utilizado variables de entorno. Esto es así debido a que, no se quería
poner en el código las claves de la aplicación de OAuth de GitHub mediante el
uso de `Hard Code'. El archivo que se ha modificado del servidor apache 2.4
se llama envvars. A continuación se muestra el contenido del mismo, donde
se sustituye el valor de las claves anteriormente citadas por puntos suspensivos
(\ldots).

\lstset{
  texcl=true,
  basicstyle=\ttfamily,
  columns=fullflexible,
  frame=single,
  breaklines=true,
  postbreak=\mbox{\textcolor{red}{$\hookrightarrow$}\space},
}
\lstinputlisting[]{envvars}

\end{document}
