\documentclass{article}
\usepackage[spanish]{babel}
\usepackage{graphicx}
\usepackage[utf8]{inputenc}
\usepackage{fancyhdr}
\usepackage{lastpage}
\usepackage{enumitem}
\usepackage{listings}

\pagestyle{fancy}
\fancyhf{}
\rfoot{Page \thepage\hspace{1pt} de~\pageref{LastPage}}

\title{Practica }
\author{Guillermo Lopez Garcia}
\begin{document}

\textbf{Ejercicio 1. Comandos.}
\begin{enumerate}
    \item Herramientas y particiones
        \begin{enumerate}
            \item Herramientas: Gparted y fdisk en linux.
            \item Particiones: yo particularmente uso 3 particiones. Son: Una para el sistema operativo, otra
                  para /home para mi datos
                  personales y otra para /etc, donde se guarda los archivos de configuracion global
                  para los programas que uso (Apache 2,
                  Nginx, MySQL, PostgreSQL, Snap, mono-xsp4 y un largo etc).
        \end{enumerate}
    \item Variable de entorno
        \begin{enumerate}
            \item La variable de entorno es PATH en GNULinux. Tengo entendido que en windows
                  tambien se llama path, pero windows
                  al ser un gestor de ventanas y no ser un sistema operativo de verdad, da igual.
        \end{enumerate}
    \item Configuracion de red
        \begin{enumerate}
            \item Se aplica la nueva configuración, y al ser GNU/Linux un sistema operativo
                  robusto y pensado para funcionar bien,
                  no como los gestores de ventanas con infulas, se usa el comando sudo service restart networking.
        \end{enumerate}
    \item Comando con sudo de utilidad
        \begin{enumerate}
            \item 1º Comando: sudo service mysql restart. Se usa para reiniciar el servicio de MySQL
            \item 2º Comando: sudo service apache2 restart. Se usa para reiniciar el servicio de Apache 2.
            \item 3º Comando: sudo apt install npm. Se usa para instalar el sistema de gestion de software npm, que se usa
                  para instalar cositas chulas de nodejs, javascript, typescript y demás.
        \end{enumerate}
\end{enumerate}

\textbf{Ejercicio 2. Gestion de Usuarios.}
\begin{enumerate}
    \item
        \begin{enumerate}
            \item whoami
            \item groups
            \item who
            \item cat /etc/group | grep sudo -n
        \end{enumerate}
    \item
        \begin{enumerate}
            \item sudo useradd -d /home/sd, sudo passwd sd, sudo mkdir /home/sd
            \item sudo groupadd gruposd, sudo usermod -a -G gruposd sd.
            \item id sd, sudo usermod -d /home/comunes -m
        \end{enumerate}
\end{enumerate}

\textbf{Ejercicio 3. Gestion del SO usando Python.}
\begin{enumerate}
    \item
        \lstset{language=Python, texcl=true}
        \begin{lstlisting}[frame=single]
            from os import environ
            Path = environ[PATH]
            print(Path)
        \end{lstlisting}
    \item
        \lstset{language=Python, texcl=true}
        \begin{lstlisting}[frame=single]
            from shutil import copyfile
            copyfile(
                "/etc/network/interfaces",
                "interfaces_bck"
            )
        \end{lstlisting}
    \item
        \lstset{language=Python, texcl=true}
        \begin{lstlisting}[frame=single]
            from filecmp import cmp
            if cmp(
                "etc/network/interfaces",
                "etc/network/interfaces_bck"
            ):
                print("True")
        \end{lstlisting}
    \item
        \lstset{language=Python, texcl=true}
        \begin{lstlisting}[frame=single]
            from os import listdir
            from pprint import pprint
            dirlist = listdir("/home")
        \end{lstlisting}
    \item
        \lstset{language=Python, texcl=true}
        \begin{lstlisting}[frame=single]
            import sys
            
            name = str(sys.argv[1])
            with open("etc/passwd") as archivo:
                for linea in archivo:
                    if name in linea:
                        cadena = linea
            
            print(cadena.split("/")[4])
        \end{lstlisting}
\end{enumerate}
\end{document}
